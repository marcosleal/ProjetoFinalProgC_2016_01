\documentclass[a4paper,10pt]{article}
\usepackage[utf8]{inputenc}
\usepackage[brazil]{babel}

\usepackage{listings}
\usepackage{listingsutf8}
\usepackage{rtsched}

%opening
\title{Diagrama de execução}
\author{Autor\_1\\
        Autor\_2}

\begin{document}

\maketitle

\begin{figure}[h]
  \centering
 
  %Cria ambiente, 3 tarefas, escala de tempo até 20
  \begin{RTGrid}[nosymbols=1,width=10cm]{3}{20}
  
  
    % Tarefa 1
    \RowLabel{1}{$\tau_1$}
    \TaskArrDead{1}{0}{4}   
    \TaskArrDead{1}{4}{4}
    \TaskArrDead{1}{8}{4}
    \TaskArrDead{1}{12}{4}
    \TaskArrDead{1}{16}{4}
    % Tarefa 1 - Execução
    \TaskExecution{1}{0}{1}
    \TaskExecution{1}{4}{5}
    \TaskExecution{1}{8}{9}
    \TaskExecution{1}{12}{13}
    \TaskExecution{1}{16}{17}

    
    
     % Tarefa 2
    \RowLabel{2}{$\tau_2$}
    \TaskArrDead{2}{0}{4}
    \TaskArrDead{2}{6}{4}
    \TaskArrDead{2}{12}{4}
     % Tarefa 2 - Execução
    \TaskExecution{2}{1}{4}
    \TaskExecution{2}{6}{8}
    \TaskExecution{2}{9}{10}
    \TaskExecution{2}{13}{16}
    
    % Utilização do processador
    \RowLabel{3}{CPU}
    \TaskExecution{3}{0}{5}
    \TaskExecution{3}{6}{10}
    \TaskExecution{3}{12}{17}    
  \end{RTGrid}
\caption{Exemplo de escalonamento para duas tarefas.}
\label{fig:ex1}
\end{figure}
\end{document}
